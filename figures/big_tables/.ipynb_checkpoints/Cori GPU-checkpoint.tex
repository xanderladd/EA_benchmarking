\begin{tabular}{lrrrrrrrrlrrrrlrlrrrrrl}
\toprule
{} &  index &  Unnamed: 0 &  Nodes &  Total Cpus &  Total Gpus &  Offspring &  Stimuli &  Score Functions &       Runtime &  Runtime Stddev &       FOM &  FOM Std Dev &  GPU Utilization & Mean Eval Time &  Std Eval Time &     Sim Time &  Time to 50 Gen &  Mean Gen Size &  Std Gen Size &  Num Trials &    expected & Experiment \\
\midrule
0 &      0 &           0 &    1.0 &        42.0 &         8.0 &      250.0 &      8.0 &             20.0 &   25.0 ± 3.47 &        3.473825 &  1.270914 &     0.151831 &              0.0 &   13.92 ± 3.49 &       3.489188 &   8.6 ± 0.59 &   337169.304187 &     200.690000 &     11.012443 &       599.0 &   19.987755 &   Cori GPU \\
1 &      2 &           2 &    2.0 &        84.0 &        16.0 &      500.0 &      8.0 &             20.0 &  24.92 ± 3.26 &        3.258016 &  1.272279 &     0.140855 &              0.0 &   13.72 ± 3.18 &       3.183778 &  8.62 ± 0.54 &  -311952.174448 &     398.390000 &     18.148771 &       599.0 &   39.975509 &   Cori GPU \\
2 &      3 &           3 &    4.0 &       168.0 &        32.0 &     1000.0 &      8.0 &             20.0 &  24.38 ± 2.84 &        2.837450 &  1.297089 &     0.131855 &              0.0 &    13.1 ± 2.76 &       2.764866 &  8.68 ± 0.51 &  -311846.207834 &     797.760000 &     33.298685 &       599.0 &   79.951018 &   Cori GPU \\
3 &      4 &           4 &    8.0 &       336.0 &        64.0 &     2000.0 &      8.0 &             20.0 &  26.05 ± 4.72 &        4.719296 &  1.232871 &     0.183724 &              0.0 &   14.43 ± 4.61 &       4.613198 &  9.19 ± 0.54 &  -311709.703291 &    1592.067568 &     71.974157 &       443.0 &  159.902036 &   Cori GPU \\
4 &      5 &           5 &   16.0 &       672.0 &       128.0 &     4000.0 &      8.0 &             20.0 &  27.88 ± 2.29 &        2.293143 &  1.127844 &     0.086365 &              0.0 &   15.63 ± 2.06 &       2.063984 &  9.89 ± 0.91 &        0.000000 &    3216.111111 &    224.486630 &       161.0 &  319.804072 &   Cori GPU \\
\bottomrule
\end{tabular}
